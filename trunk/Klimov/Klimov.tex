\documentclass[12pt,a4paper]{report}
\usepackage[utf8]{inputenc}
\usepackage[russian]{babel}
\usepackage[OT1]{fontenc}
\usepackage{amsmath}
\usepackage{amsfonts}
\usepackage{amssymb}
\usepackage{graphicx}
\usepackage{cmap}					% поиск в PDF
\usepackage{mathtext} 				% русские буквы в формулах
%\usepackage{tikz-uml}               % uml диаграммы

% TODOs
\usepackage[%
  colorinlistoftodos,
  shadow
]{todonotes}

% Генератор текста
\usepackage{blindtext}

%------------------------------------------------------------------------------

% Подсветка синтаксиса
\usepackage{color}
\usepackage{xcolor}
\usepackage{listings}
 
 % Цвета для кода
\definecolor{string}{HTML}{B40000} % цвет строк в коде
\definecolor{comment}{HTML}{008000} % цвет комментариев в коде
\definecolor{keyword}{HTML}{1A00FF} % цвет ключевых слов в коде
\definecolor{morecomment}{HTML}{8000FF} % цвет include и других элементов в коде
\definecolor{captiontext}{HTML}{FFFFFF} % цвет текста заголовка в коде
\definecolor{captionbk}{HTML}{999999} % цвет фона заголовка в коде
\definecolor{bk}{HTML}{FFFFFF} % цвет фона в коде
\definecolor{frame}{HTML}{999999} % цвет рамки в коде
\definecolor{brackets}{HTML}{B40000} % цвет скобок в коде
\DeclareGraphicsExtensions{.pdf,.png,.jpg}

 % Настройки отображения кода
\lstset{
language=C, % Язык кода по умолчанию
morekeywords={*,...}, % если хотите добавить ключевые слова, то добавляйте
 % Цвета
keywordstyle=\color{keyword}\ttfamily\bfseries,
stringstyle=\color{string}\ttfamily,
commentstyle=\color{comment}\ttfamily\itshape,
morecomment=[l][\color{morecomment}]{\#}, 
 % Настройки отображения     
breaklines=true, % Перенос длинных строк
basicstyle=\ttfamily\footnotesize, % Шрифт для отображения кода
backgroundcolor=\color{bk}, % Цвет фона кода
%frame=lrb,xleftmargin=\fboxsep,xrightmargin=-\fboxsep, % Рамка, подогнанная к заголовку
frame=tblr
rulecolor=\color{frame}, % Цвет рамки
tabsize=3, % Размер табуляции в пробелах
 % Настройка отображения номеров строк. Если не нужно, то удалите весь блок
numbers=left, % Слева отображаются номера строк
stepnumber=1, % Каждую строку нумеровать
numbersep=5pt, % Отступ от кода 
numberstyle=\small\color{black}, % Стиль написания номеров строк
 % Для отображения русского языка
extendedchars=true,
literate={Ö}{{\"O}}1
  {Ä}{{\"A}}1
  {Ü}{{\"U}}1
  {ß}{{\ss}}1
  {ü}{{\"u}}1
  {ä}{{\"a}}1
  {ö}{{\"o}}1
  {~}{{\textasciitilde}}1
  {а}{{\selectfont\char224}}1
  {б}{{\selectfont\char225}}1
  {в}{{\selectfont\char226}}1
  {г}{{\selectfont\char227}}1
  {д}{{\selectfont\char228}}1
  {е}{{\selectfont\char229}}1
  {ё}{{\"e}}1
  {ж}{{\selectfont\char230}}1
  {з}{{\selectfont\char231}}1
  {и}{{\selectfont\char232}}1
  {й}{{\selectfont\char233}}1
  {к}{{\selectfont\char234}}1
  {л}{{\selectfont\char235}}1
  {м}{{\selectfont\char236}}1
  {н}{{\selectfont\char237}}1
  {о}{{\selectfont\char238}}1
  {п}{{\selectfont\char239}}1
  {р}{{\selectfont\char240}}1
  {с}{{\selectfont\char241}}1
  {т}{{\selectfont\char242}}1
  {у}{{\selectfont\char243}}1
  {ф}{{\selectfont\char244}}1
  {х}{{\selectfont\char245}}1
  {ц}{{\selectfont\char246}}1
  {ч}{{\selectfont\char247}}1
  {ш}{{\selectfont\char248}}1
  {щ}{{\selectfont\char249}}1
  {ъ}{{\selectfont\char250}}1
  {ы}{{\selectfont\char251}}1
  {ь}{{\selectfont\char252}}1
  {э}{{\selectfont\char253}}1
  {ю}{{\selectfont\char254}}1
  {я}{{\selectfont\char255}}1
  {А}{{\selectfont\char192}}1
  {Б}{{\selectfont\char193}}1
  {В}{{\selectfont\char194}}1
  {Г}{{\selectfont\char195}}1
  {Д}{{\selectfont\char196}}1
  {Е}{{\selectfont\char197}}1
  {Ё}{{\"E}}1
  {Ж}{{\selectfont\char198}}1
  {З}{{\selectfont\char199}}1
  {И}{{\selectfont\char200}}1
  {Й}{{\selectfont\char201}}1
  {К}{{\selectfont\char202}}1
  {Л}{{\selectfont\char203}}1
  {М}{{\selectfont\char204}}1
  {Н}{{\selectfont\char205}}1
  {О}{{\selectfont\char206}}1
  {П}{{\selectfont\char207}}1
  {Р}{{\selectfont\char208}}1
  {С}{{\selectfont\char209}}1
  {Т}{{\selectfont\char210}}1
  {У}{{\selectfont\char211}}1
  {Ф}{{\selectfont\char212}}1
  {Х}{{\selectfont\char213}}1
  {Ц}{{\selectfont\char214}}1
  {Ч}{{\selectfont\char215}}1
  {Ш}{{\selectfont\char216}}1
  {Щ}{{\selectfont\char217}}1
  {Ъ}{{\selectfont\char218}}1
  {Ы}{{\selectfont\char219}}1
  {Ь}{{\selectfont\char220}}1
  {Э}{{\selectfont\char221}}1
  {Ю}{{\selectfont\char222}}1
  {Я}{{\selectfont\char223}}1
  {і}{{\selectfont\char105}}1
  {ї}{{\selectfont\char168}}1
  {є}{{\selectfont\char185}}1
  {ґ}{{\selectfont\char160}}1
  {І}{{\selectfont\char73}}1
  {Ї}{{\selectfont\char136}}1
  {Є}{{\selectfont\char153}}1
  {Ґ}{{\selectfont\char128}}1
  {\{}{{{\color{brackets}\{}}}1 % Цвет скобок {
  {\}}{{{\color{brackets}\}}}}1 % Цвет скобок }
}
 
 % Для настройки заголовка кода
\usepackage{caption}
\DeclareCaptionFont{white}{\color{сaptiontext}}
\DeclareCaptionFormat{listing}{\parbox{\linewidth}{\colorbox{сaptionbk}{\parbox{\linewidth}{#1#2#3}}\vskip-4pt}}
\captionsetup[lstlisting]{format=listing,labelfont=white,textfont=white}
\renewcommand{\lstlistingname}{Код} % Переименование Listings в нужное именование структуры


%------------------------------------------------------------------------------

\author{С.~А.~Климов}
\title{Сети ЭВМ и телекоммуникации}
\begin{document}
%\listoftodos
\maketitle
\chapter{Задание}
Разработать приложение-клиент и приложение сервер электронной почты.
\section{Функциональные требования}
%\todo[inline]{Начать можно с этого}
Серверное приложение должно реализовывать следующие функции:
\begin{enumerate}
\item Прослушивание определенного порта
\item Обработка запросов на подключение по этому порту от клиентов
\item Поддержка одновременной работы нескольких почтовых клиентов
через механизм нитей
\item Приём почтового сообщения от одного клиента для другого
\item Хранение электронной почты для клиентов
\item Посылка клиенту почтового сообщения по запросу с последующим
удалением сообщения
\item Посылка клиенту сведений о состоянии постового ящика
\item Обработка запроса на отключение клиента
\item Принудительное отключение клиента
\end{enumerate}
Клиентское приложение должно реализовывать следующие функции:
\begin{enumerate}
\item Установление соединения с сервером
\item Передача электронного письма на сервер для другого клиента
\item Проверка состояния своего почтового ящика
\item Получение конкретного письма с сервера
\item Разрыв соединения
\item Обработка ситуации отключения клиента сервером
\end{enumerate}
\section{Нефункциональные требования}
Требования к производительности, надежности, целевым платформам и~т.п.
Серверное приложение должно работать одновременно с 5-ю клиентами. Также приложение должно выдвать клиенту список его возможных дальнейших операций(чтение сообщений, отправка сообщения). Серверное приложение должно работать 2 часа без перерывов, обслуживая всех, поключенных к нему клиентов(в пределах заданного ограничения). Приложение должно обслуживать клиентов на платформах Linux и Wondows.


Разработанное клиентское приложение должно не выходить из строя при отправке сообщения.
\section{Накладываемые ограничения}
Имя пользователя не должно превышать 10-ти символов.
Сообщение не дожно превышать 230 символов.
Одновременное количесво подключенных к серверу клиентов не превышает 5.
\chapter{Реализация для работы по протоколу TCP}
\section{Прикладной протокол}

\label{protocol_tcp}
\begin{tabular}{|c|c|}
\hline 
Первая команда ввода имя пользователя & username\# \\ 
\hline 
Команда написать kitty сщщбщение hello! & 1\#kitty\#hello!\# \\ 
\hline 
Команда прочитать входящие & 2\# \\
\hline 
Команда выхода & 3\# \\
\hline 
\end{tabular} 

\section{Архитектура приложения} 
Взаимодействие сервера и клиента начинается с отправки клиентом сообщение с его логином. Так сервер идентифицирует его в системе. Затем клиент посылает серверу разлчиные команды, он может отправить сообщение, прочитать свою почту и выйти.
\linebreak
\center \includegraphics[scale=0.8]{TCP_comm.jpg}
\flushleft
%\begin{figure}[p]
 %   \includegraphics{TCP_comm.jpg}
%\end{figure}

\section{Тестирование}
\subsection{Описание тестового стенда и методики тестирования}
Тестирование проводилось на виртуальной машине Debian 4.7. Было запущено приложение сервера, затем - несколько приложений клента. Таким образом сервер и клиент работали на одном компьютере.
\subsection{Тестовый план и результаты тестирования}
По шагам, с перечнем входных данных
На первом клиенте был осуществлен вход под логином "serg", было отправлнео сообщение клиенту "ali". Был получен список входящих сообщений для пользователя "serg". В другом терминале был запущен клиент с именем "ali"(терминал пользователя "serg" оставался активным), далее был получен для него список входящих сообщений, последним из них оказалось только что отправленное сообщение пользователем "serg". Было отправлено сообщение пользователю "serg" и осуществлен выход. На терминале пользователя "serg" были прочитаны сообщения, последним из них оказалось только что отправленное сообщение пользователем "ali". Был осуществлен выход.
При вводе некорректных команд клиентское приложение сообщает об этом пользователю и продолжает работу. Серверное приложение так же выводит на консоль не корректные команды от пользователя.
\chapter{Реализация для работы по протоколу UDP}
\section{Прикладной протокол}

Прикладной протокол не претерпел изменений по сравнению с пунктом аналогичным для TCP(см. п.\ref{protocol_tcp}).

\section{Архитектура приложения}
Взаимодействие сервера и клиента начинается с отправки клиентом приветственного сообщение, по этому сообщению сервер отсылает в ответ строку с предложением ввести свой логин.Клиент отправляет свой логин. Так сервер идентифицирует его в системе. Сервер отсылает строку с возможными командыми. Затем клиент посылает серверу разлчиные команды, он может отправить сообщение, прочитать свою почту и выйти.
\linebreak
\center \includegraphics[scale=0.8]{UDP_comm.jpg}
\flushleft

\section{Тестирование}
\subsection{Описание тестового стенда и методики тестирования}
Тестирование проводилось на виртуальной машине Debian 4.7. Было запущено приложение сервера, затем - несколько приложений клента. Таким образом сервер и клиент работали на одном компьютере.
\subsection{Тестовый план и результаты тестирования}
По шагам, с перечнем входных данных
На первом клиенте был осуществлен вход под логином "serg", было отправлнео сообщение клиенту "ali". Был получен список входящих сообщений для пользователя "serg". В другом терминале был запущен клиент с именем "ali"(терминал пользователя "serg" оставался активным), далее был получен для него список входящих сообщений, последним из них оказалось только что отправленное сообщение пользователем "serg". Было отправлено сообщение пользователю "serg" и осуществлен выход. На терминале пользователя "serg" были прочитаны сообщения, последним из них оказалось только что отправленное сообщение пользователем "ali". Был осуществлен выход.
При вводе некорректных команд клиентское приложение сообщает об этом пользователю и продолжает работу. Серверное приложение так же выводит на консоль не корректные команды от пользователя.
\chapter{Выводы}
Анализ выполненных заданий, сравнение удобства/эффективности/количества проблем при программировании TCP/UDP
\section{Реализация для TCP}
Когда взаимодействие осуществляется через TCP, обеспечивается надежная передача потока байтов как от приложения сервера к приложению клиента, так и в обратном направлении. Также использование TCP удобно, потому что осуществляется контроль длины сообщения, скорость обмена сообщениями и сетевой трафик средствами самого протокола TCP. 
\section{Реализация для UDP}
Когда взаимодействие осуществляется через UDP, не применяется модель взаимодействия клиента и сервера, использующая так называемые "неявные" рукопожатия. Из-за этого не осуществляется упорядочивание и контроль целостности данных средсвтами самого протокола, это приходится реализовывать вручную. Использование UDP накладывает на разработчка ряд дополнительных задач, такие как контроль целостности данных и их упорядочивание.
\linebreak
\linebreak
При дальнейшей разработке сетевых приложений я бы использовал протокол TCP, т.к. он удобнее и легче в реализации.

\chapter{Приложения}
\section{Описание среды разработки}
Версии ОС, компиляторов, утилит, и проч., которые использовались в процессе разработки
\section{Листинги}
\subsection{Сервер TCP для Linux. Основной файл программы main.c}
\lstinputlisting{./Server1/main.c}
%{/home/user/workspace/tcp_server/main.c}
\subsection{Сервер TCP для Linux. Файл сборки Makefile}
\lstinputlisting{./Server1/Makefile.make}
\subsection{Клиент TCP для Linux. Основной файл программы main.c}
\lstinputlisting{./Client1/main.c}
%{/home/user/workspace/tcp_server/main.c}
\subsection{Клиент TCP для Linux. Файл сборки Makefile}
\lstinputlisting{./Client1/Makefile.make}
\subsection{Сервер UDP для Linux. Основной файл программы main.c}
\lstinputlisting{./Server3/main.c}
\subsection{Сервер UDP для Linux. Файл сборки Makefile}
\lstinputlisting{./Server3/Makefile.make}
\subsection{Клиент UDP для Linux. Основной файл программы main.c}
\lstinputlisting{./Client3/main.c}
%{/home/user/workspace/tcp_server/main.c}
\subsection{Клиент UDP для Linux. Файл сборки Makefile}
\lstinputlisting{./Client3/Makefile.make}
\subsection{Сервер TCP для Windows. Основной файл программы main.c}
\lstinputlisting{./serv1_win/serv1_win/main.c}
\subsection{Клиент TCP для Windows. Основной файл программы main.c}
\lstinputlisting{./clie1_win/clie1_win/main.c}
%\lstinputlisting[language=make,label={Makefile}]
%{/home/user/workspace/tcp_server/Makefile}
%\todo[inline]{Не забыть вставить все исходники}
\end{document}