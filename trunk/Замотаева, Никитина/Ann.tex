\documentclass[10pt,a4paper]{report}
\usepackage[utf8]{inputenc}
\usepackage[russian]{babel}
\usepackage[OT1]{fontenc}
\usepackage{amsmath}
\usepackage{amsfonts}
\usepackage{amssymb}
\usepackage{graphicx}
\author{Никитина А.В., Замотаева Ю.И.}
\title{Отчет по лабораторной работе №1 по дисциплине: "Телекоммуникационные системы и сети"\newline
Тема: "Ознакомление с системой \TeX\ и расширением \LaTeX "}
\date{13 марта 2014}
\begin{document}
\maketitle
\pagebreak
\chapter{Введение}
\section{Цель работы}
Ознакомление с программой для верстки отчетов \TeX.
\section{Задача работы}
Научиться создавать титульный лист, разделы, списки и писать несложные формулы в среде верстки отчетов \TeX.
\chapter{Ход работы}
\section{Написание математических формул}
\begin{enumerate}
\item
\begin{displaymath}
lim_{n \to \infty}
\sum_{k=1}^n \frac{25}{2*k^3}
= \frac{\pi}{2}
\end{displaymath}
\item 
\begin{displaymath}
x^{3}+\sum_{x=1}^8 \frac{x}{2}
= \sqrt{617}
\end{displaymath}
\item 
\begin{displaymath}
x(t) = A * sin(w*t+f)
\end{displaymath}
\end{enumerate}
Формула интеграла:
\begin{displaymath}
\iint_{\frac{sin(x)}{x}+y^{2} = 1} f(x, y) dx dy 
\end{displaymath}
\section{Интерпретация результатов}
Среда верстки отчетов \TeX очень удобна для введения математических формул. Имеется множество команды, которые заметно облегчают ввод дробных выражений, констант, переменных, производных, интегралов, дифференциалов и других формул. Таким образом, можно без особы трудностей вводить формулы различной сложности.
\chapter{Вывод}
В данной лабораторной работе мы познакомились  с системой создания и редактирования текстов \TeX и расширением \LaTeX. Рассмотрели команды этой среды, которые помогают составлять красивые и аккуратные отчеты.
Таким образом, теперь мы без труда может написать отчет, который будет иметь титульный лист, главы, заголовки и различные математические формулы. Нам очень понравилась эта система, потому что она помогает каждому составить отчет быстро и удобно. Система полностью автоматизирована, от нас требуется только освоить основные команды и ознакомиться с дополнительной литературой.
\end{document}