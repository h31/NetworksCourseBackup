\documentclass[10pt,a4paper]{report}
\usepackage[utf8]{inputenc}
\usepackage[russian]{babel}
\usepackage[OT1]{fontenc}
\usepackage{amsmath}
\usepackage{amsfonts}
\usepackage{amssymb}
\usepackage{graphicx}
\author{Кенть Н.В.}
\title{Отчет по лабораторным работам по дисциплине Телекоммуникационные системы и сети\newline
на тему "Ознакомление с системой верски \TeX\ и расширением \LaTeX "}
\date{Февраль 2014}
\begin{document}
\maketitle
\pagebreak
\chapter{Ознакомление}
\section{Цель работы}
Изучение принципов работы \TeX и создание первого отчета
\section{Написание формул}
Важная задача - научиться писать формулы в системе. Изучив этот аспект системы, создание отчетов и презентаций  станет гораздо удобнее.


\chapter{Ход работы}
\section{Написание формул}
\begin{displaymath}
X(t) = A_0 * cos(2*t\pi)
\end{displaymath}
\begin{displaymath}
f(x) = \frac{A_0}{2} + \sum \limits_{n=1}^{\infty} A_n \cos \left( \frac{2 n \pi x}{\nu} - \alpha_n \right)
\end{displaymath}
\begin{displaymath}
c^2 = a^2 + b^2 \frac{a+b}{c-d} \sum_{i=0}^{N}k _i
\end{displaymath}
Формула производных:
\begin{displaymath}
dz = \frac{\partial z}{\partial x} dx + \frac{\partial z}{\partial y} dy
\end{displaymath}
Формула интеграла:
\begin{displaymath}
\iint_{x^2 + y^9 = 1} f(x, y) dx dy 
\end{displaymath}

\begin{displaymath}
f(x,y,\alpha, \beta) = \frac{\sum \limits_{n=1}^{\infty} 
A_n \cos \left( \frac{2 n \pi x}{\nu} \right)} {\prod \mathcal{F} {g(x,y)} }-\int{x} f(x) dx
\end{displaymath}
\section{Вывод}
В данной работе мы ознакомились  с системой верски \TeX\ и расширением \LaTeX. Подробно узнали принципы создания отчетов в этой системе. Главный, на мой взгляд, плюс - это удобство создания отчетов, при минимальных навыках создание отчетности превращается из рутинной, нудной работы в достаточно приятную деятельность, когда не нужно постоянно поправлять съехавший текст, кривую формулу. Минус для меня - неизвестная среда, нужно знать синтаксис, который достаточно прост и быстро учится. Отличная программа для облегчения жизни!
\end{document}

