\documentclass[10pt,a4paper]{report}
\usepackage[utf8]{inputenc}
\usepackage[russian]{babel}
\usepackage[OT1]{fontenc}
\usepackage{amsmath}
\usepackage{amsfonts}
\usepackage{amssymb}
\usepackage{graphicx}
\author{Киселёв А.А.}
\title{Отчет по лабораторной работе №1 по дисциплине: "Телекоммуникационные системы и сети"\newline
Тема: "Ознакомление с системой \TeX\ и расширением \LaTeX "}
\date{28 февраля 2014}
\begin{document}
\maketitle
\pagebreak
\chapter{Введение}
\section{Цель работы}
Изучение принципов работы \TeX и создание первого отчета.
\section{Задача работы}
Научиться писать математические формулы и выражения в среде создания текстов \TeX.
\chapter{Ход работы}
\section{Создание математической формулы}
\begin{displaymath}
x(t) = A * cos(w*t+f)
\end{displaymath}
\begin{displaymath}
F(x) = \frac{A}{2} + \sum \limits_{i=1}^{\infty} A_i \sin \left( \frac{2 i \pi x}{\nu} + \beta_i \right)
\end{displaymath}
\begin{displaymath}
\delta = a^4 + \frac{a}{c-d^2} \sum_{i=0}^{N}(k _i^3+4*k_i^2-7*k_i+19) + f_0 *  \sum_{i=0}^{N}(beta_i^2)
\end{displaymath}
Формула производной от интеграла:
\begin{displaymath}
\frac{d}{dx}\int F(x) dx=F(x)
\end{displaymath}
Формула интеграла:
\begin{displaymath}
\iiint_{x^2 + y^2 + z^2 = 1} f(x, y, z) dx dy dz 
\end{displaymath}
\section{Интерпретация результатов}
Для введения математических формул система обладает множеством инструментов. Как видим по результатам имеется набор операторов, позволяющих вводить дробные выражения, выражения в скобках, константы, переменные, суммы, производные, дифференциалы, интегралы и др. 
Система располагает возможностями составления сколь угодно больших выражений.
\chapter{Вывод}
В данной лабораторной работе мы познакомились  с системой создания и редактирования текстов \TeX\ и расширением \LaTeX. Освоили простейшие конструкции для составления и написания красивых отчетов, таких как оформление титульного листа, глав, заголовков и перечислений, особенно мы изучили способы введения и построения математических формул и выражений. Безусловными плюсами я считаю автоматизированное составление отчетов с использованием логики алгоритмов, которые заменяют ручное редактирование и написание текста. Система позволяет грамотно и аккуратно строить текстовые конструкции. Минусом можно считать лишь то, что система требует освоения своего собственного синтаксиса.
\end{document}