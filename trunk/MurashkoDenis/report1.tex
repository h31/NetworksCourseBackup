\documentclass[12pt,a4paper]{report}
\usepackage[utf8]{inputenc}
\usepackage[russian]{babel}
\usepackage[OT1]{fontenc}
\usepackage{amsmath}
\usepackage{amsfonts}
\usepackage{amssymb}
\usepackage{graphicx}
%\graphicspath{{pictures/}}
%\DeclareGraphicsExtensions{.pdf,.png,.jpg}
\usepackage{cmap}					% поиск в PDF
\usepackage{mathtext} 				% русские буквы в формулах



% Генератор текста
\usepackage{blindtext}

%------------------------------------------------------------------------------

% Подсветка синтаксиса
\usepackage{color}
\usepackage{xcolor}
\usepackage{listings}
 
 % Цвета для кода
\definecolor{string}{HTML}{B40000} % цвет строк в коде
\definecolor{comment}{HTML}{008000} % цвет комментариев в коде
\definecolor{keyword}{HTML}{1A00FF} % цвет ключевых слов в коде
\definecolor{morecomment}{HTML}{8000FF} % цвет include и других элементов в коде
\definecolor{captiontext}{HTML}{FFFFFF} % цвет текста заголовка в коде
\definecolor{captionbk}{HTML}{999999} % цвет фона заголовка в коде
\definecolor{bk}{HTML}{FFFFFF} % цвет фона в коде
\definecolor{frame}{HTML}{999999} % цвет рамки в коде
\definecolor{brackets}{HTML}{B40000} % цвет скобок в коде
 
 % Настройки отображения кода
\lstset{
language=C, % Язык кода по умолчанию
morekeywords={*,...}, % если хотите добавить ключевые слова, то добавляйте
 % Цвета
keywordstyle=\color{keyword}\ttfamily\bfseries,
stringstyle=\color{string}\ttfamily,
commentstyle=\color{comment}\ttfamily\itshape,
morecomment=[l][\color{morecomment}]{\#}, 
 % Настройки отображения     
breaklines=true, % Перенос длинных строк
basicstyle=\ttfamily\footnotesize, % Шрифт для отображения кода
backgroundcolor=\color{bk}, % Цвет фона кода
%frame=lrb,xleftmargin=\fboxsep,xrightmargin=-\fboxsep, % Рамка, подогнанная к заголовку
frame=tblr
rulecolor=\color{frame}, % Цвет рамки
tabsize=3, % Размер табуляции в пробелах
 % Настройка отображения номеров строк. Если не нужно, то удалите весь блок
numbers=left, % Слева отображаются номера строк
stepnumber=1, % Каждую строку нумеровать
numbersep=5pt, % Отступ от кода 
numberstyle=\small\color{black}, % Стиль написания номеров строк
 % Для отображения русского языка
extendedchars=true,
literate={Ö}{{\"O}}1
  {Ä}{{\"A}}1
  {Ü}{{\"U}}1
  {ß}{{\ss}}1
  {ü}{{\"u}}1
  {ä}{{\"a}}1
  {ö}{{\"o}}1
  {~}{{\textasciitilde}}1
  {а}{{\selectfont\char224}}1
  {б}{{\selectfont\char225}}1
  {в}{{\selectfont\char226}}1
  {г}{{\selectfont\char227}}1
  {д}{{\selectfont\char228}}1
  {е}{{\selectfont\char229}}1
  {ё}{{\"e}}1
  {ж}{{\selectfont\char230}}1
  {з}{{\selectfont\char231}}1
  {и}{{\selectfont\char232}}1
  {й}{{\selectfont\char233}}1
  {к}{{\selectfont\char234}}1
  {л}{{\selectfont\char235}}1
  {м}{{\selectfont\char236}}1
  {н}{{\selectfont\char237}}1
  {о}{{\selectfont\char238}}1
  {п}{{\selectfont\char239}}1
  {р}{{\selectfont\char240}}1
  {с}{{\selectfont\char241}}1
  {т}{{\selectfont\char242}}1
  {у}{{\selectfont\char243}}1
  {ф}{{\selectfont\char244}}1
  {х}{{\selectfont\char245}}1
  {ц}{{\selectfont\char246}}1
  {ч}{{\selectfont\char247}}1
  {ш}{{\selectfont\char248}}1
  {щ}{{\selectfont\char249}}1
  {ъ}{{\selectfont\char250}}1
  {ы}{{\selectfont\char251}}1
  {ь}{{\selectfont\char252}}1
  {э}{{\selectfont\char253}}1
  {ю}{{\selectfont\char254}}1
  {я}{{\selectfont\char255}}1
  {А}{{\selectfont\char192}}1
  {Б}{{\selectfont\char193}}1
  {В}{{\selectfont\char194}}1
  {Г}{{\selectfont\char195}}1
  {Д}{{\selectfont\char196}}1
  {Е}{{\selectfont\char197}}1
  {Ё}{{\"E}}1
  {Ж}{{\selectfont\char198}}1
  {З}{{\selectfont\char199}}1
  {И}{{\selectfont\char200}}1
  {Й}{{\selectfont\char201}}1
  {К}{{\selectfont\char202}}1
  {Л}{{\selectfont\char203}}1
  {М}{{\selectfont\char204}}1
  {Н}{{\selectfont\char205}}1
  {О}{{\selectfont\char206}}1
  {П}{{\selectfont\char207}}1
  {Р}{{\selectfont\char208}}1
  {С}{{\selectfont\char209}}1
  {Т}{{\selectfont\char210}}1
  {У}{{\selectfont\char211}}1
  {Ф}{{\selectfont\char212}}1
  {Х}{{\selectfont\char213}}1
  {Ц}{{\selectfont\char214}}1
  {Ч}{{\selectfont\char215}}1
  {Ш}{{\selectfont\char216}}1
  {Щ}{{\selectfont\char217}}1
  {Ъ}{{\selectfont\char218}}1
  {Ы}{{\selectfont\char219}}1
  {Ь}{{\selectfont\char220}}1
  {Э}{{\selectfont\char221}}1
  {Ю}{{\selectfont\char222}}1
  {Я}{{\selectfont\char223}}1
  {і}{{\selectfont\char105}}1
  {ї}{{\selectfont\char168}}1
  {є}{{\selectfont\char185}}1
  {ґ}{{\selectfont\char160}}1
  {І}{{\selectfont\char73}}1
  {Ї}{{\selectfont\char136}}1
  {Є}{{\selectfont\char153}}1
  {Ґ}{{\selectfont\char128}}1
  {\{}{{{\color{brackets}\{}}}1 % Цвет скобок {
  {\}}{{{\color{brackets}\}}}}1 % Цвет скобок }
}
 
 % Для настройки заголовка кода
\usepackage{caption}
\DeclareCaptionFont{white}{\color{сaptiontext}}
\DeclareCaptionFormat{listing}{\parbox{\linewidth}{\colorbox{сaptionbk}{\parbox{\linewidth}{#1#2#3}}\vskip-4pt}}
\captionsetup[lstlisting]{format=listing,labelfont=white,textfont=white}
\renewcommand{\lstlistingname}{Код} % Переименование Listings в нужное именование структуры


%------------------------------------------------------------------------------

\author{Д.~С.~Мурашко}
\title{Сети ЭВМ и телекоммуникации}
\begin{document}
%\listoftodos
\maketitle
\chapter{Калькулятор}
Разработать  приложение-сервер «Удаленный  калькулятор», 
позволяющее  по  запросу  выполнять  математические  операции,  и  удаленный клиент для сервера. 
\section{Функциональные требования}
Серверное приложение должно реализовывать следующие функции:
\begin{enumerate}
\item{Приём «быстрых» операций с аргументами от клиента. Должны поддерживаться  следующие  операции:  сложение,  вычитание,  умножение, деление. }
\item{Вычисление «долгих»  математических  операций (факториал,  квадратный  корень)  с  последующей  отложенной  посылкой  результата
клиенту(отдельная операция, инициируемая сервером).}
\item{Обработка запроса на отключение клиента}
\item{Принудительное отключение клиента}
\end{enumerate}

Клиентское приложение должно реализовывать следующие функции:
\begin{enumerate}
\item{Посылка операции с аргументами на вычисление}
\item{Получение результата вычислений«быстрых» операций}
\item{Получения результата вычислений«долгих» операций}
\item{Разрыв соединения}
\item{Обработка ситуации отключения клиента сервером}
\end{enumerate}

\section{Нефункциональные требования}
Серверное приложение:
\begin{enumerate}
\item{Прослушивание определенного порта}
\item{Обработка запросов на подключение по этому порту от клиентов}
\item{Поддержка одновременной работы нескольких клиентов через механизм нитей}
\end{enumerate}

Клиентское приложение должно реализовывать следующие функции:
\begin{enumerate}
\item{Установление соединения с сервером}
\end{enumerate}

\section{Накладываемые ограничения}
\begin{enumerate}
\item{Ограничение на вводимые данные: результат выражения может "выйти" за пределы типа double (больше 19 разрядов), причем такая ситуация возможна для всех операций. Это является большим минусов приложения. Решить эту проблему возможно с помощью вычислений с произвольной точностью, использовав, к примеру, библиотеку GMP.}
\item{Ограничения на длину сообщения: максимальный размер 100 символов.Для правильного чтения/отправки сообщения требуется фиксированная длина. Длины пакета 100 символов - достаточно для данного приложения}
\item{Ограниченние, накладываемое на вычисление факториала: аргумент не должен быть больше 16 (идет переполнение long int)}
\item {Обрыв сессии - некорректное завершение работы клиентом. При некорректном завершении сессии клиентом он останется в состоянии "online", что является минусом данного протокола}
\end{enumerate}
\chapter{Реализация для работы по протоколу TCP}
\section{Прикладной протокол}
\label{protocol_tcp}
Клиент и сервер обмениваются сообщениями. Сообщение от клиента – запрос, сообщение от сервера – ответ. В ходе работы клиент посылает запросы серверу. Сервер обязан отправить ответ. На каждый запрос должен быть отправлен только один ответ. В качестве сообщения передается строка с арифметическим выражением. Предусмотренны следующие команды: сложение, вычитание, умножение, деление, вычисление факториала и квадратного корня:
\begin{enumerate}
\item{Формат для операций сложения, вычитания, умножения и деления:  }\\
\begin{enumerate}
\item{Первое число}\\
\item{Знак операции (+, -, *, /)}\\
\item{Второе число}\\
\item{Равно (=)}\\
\end{enumerate}
Дробная часть вводится через "."\\
\item{Формат для операций факториал и квадратный корень:} \\
\begin{enumerate}
\item{Число} \\
\item{Знак операции} (!)\\
\end{enumerate}
\end{enumerate}
\section{Архитектура приложения}
\subsection{Описание клиента}
Клиент  протокола TCP  создаёт  экземпляр  сокета,  необходимый  для
взаимодействия  с  сервером,  организует  соединение,  осуществляет  обмен
данными, в соответствии с протоколом прикладного уровня. Cтруктура TCP-клиента представлена ниже:\\
\begin{figure}[htb]
\centering
\includegraphics[scale=.7]{png4.eps}
\end{figure}
%\centering \includegraphics[scale=1]{klient.jpg}\\

\subsection{Описание сервера}
Организация ТСP-сервера отличается отTCP-клиента в первую очередь
созданием  слушающего сокета. Такой сокет находится в состоянии listen  и  предназначен  только  для  приёма  входящих  соединений.  В случае  прихода  запроса  на  соединение  создаётся  дополнительный  сокет,  который  и  занимается  обменом  данными  с  клиентом.  Типичная  структура TCP-сервера и взаимосвязь сокетов изображена ниже:\\
\begin{figure}[htb]
\centering
\includegraphics[scale=.7]{png5.eps}
\end{figure}
%\[height=60mm, width=100mm]

\subsection{Дизайн приложения}
После запуска сервер начинает прослушивать определенный порт (5001) и при подключении клиента, выделяет ему отдельный поток. При подключнии к серверу выводится:\\
Input Expression:\\
Далее клиент вводит выражение согласно протоколу. В случае ввода "быстрых" операций 
клиент сразу получает ответ в формате:\\
Answer = 45 и предлогается ввести следующее выражение.\\
Результат вычисления "долгих" математических  операций (факториал,  квадратный  корень) посылаюся с  отложенной  посылкой  результата клиенту.
Для получения результата клиент вводит
команду CHECK-F или CHECK-S для факториала или квадратного корня соответствено.
Чтобы завершить соединение клиент должен ввести: q.
\subsection{Многопоточное взаимодействие с несколькими клиентами}
Для осуществления многоклиентского взаимодействия используются потоки, каждый поток защищен от другого клиента. Сервер при постоянном режиме прослушивает необходимый порт и при подключении клиента, выделяет ему отдельный поток и ждет сообщения от клиента. Поток завершается при отключении клиента.
Сервер в любой момент может отключить подключившегося к нему клиента. Для этого требуется ввести команду:\\
k[numb],\\
где [numb]-номер подключившегося клиента.
Сервер закроет поток, соответствующий этому клиенту, тем самым отключит выбранного клиента.
\section*{Описание среды разработки}
Linux debian 3.2.0-4-486 1 Debian 3.2.60-1+deb7u3 i686 GNU/Linux.\newline Среда разработки - Eclipse.\newline
Windows 7.\newline Среда разработки - Visual Studio 2010.
\section{Тестирование}
\subsection{Описание тестового стенда и методики тестирования}
Для тестирования приложения запускается сервер и несколько клиентов. В процессе тестирования проверяются основные возможности сервера по параллельному тестированию нескольких пользователей.
\subsection{Тестовый план и результаты тестирования}
Протесируем вначале быстрые операции: сложение, вычиание, умножение и деление. Проверим работу с положительными и отрицательными числами, деление на 0.\\
Input Expression:\\
2+2=\\
Answer = 4.000000\\
\\
Input Expression:\\
5-6=\\
Answer = -1.000000\\
\\
Input Expression:\\
-5+4=\\
Answer = -1.000000\\
\\
Input Expression:\\
253.4678*4515=\\
Answer = 1144407.117000\\
\\
Input Expression:\\
362626/57483435=\\
Answer = 0.006308\\
\\
Input Expression:\\
3435/0=\\
Error, input divider != 0 !\\
\\
Input Expression:\\
-56/-78=\\
Answer = 0.717949\\

Теперь проверим долгие операции: факториал и квадратный корнь. Проверим работу, как поведет себя приложение при задании неверных аргументов для этих операций, а также протестируем отложенную посылку результата. \\
строку ввести
Input Expression:\\
4!\\
Wait\\
\\
Input Expression:\\
45-90=\\
Answer = -45.000000\\
\\
Input Expression:\\
CHECK-F\\
Factorial = 24\\
\\
Input Expression:\\
45#\\
Wait\\
\\
Input Expression:\\
57/89=\\
Answer = 0.640449\\
\\
Input Expression:\\
CHECK-S\\
Sqrt = 6.708204\\
\\
Input Expression:\\
-5!\\
Error, input Factorial > 0 !\\
\\
Input Expression:\\
-5#\\
Error, input Root >= 0 !\\
\\
При задании достаточно большого аргумента для факториала, приложение "падает", т.к.
результат операции выходит из максимального значения типа long int
Input Expression:\\
20!\\
Wait\\
\\
Input Expression:\\
CHECK-F\\
Factorial = -2102132736\\

Проверим многопоточность - запустим несколько клиентов одновременно (проверялось 4 клиента одновременно, но максимально возможно 10). Приложение работает корректно со всеми клиентами.\\
Проверим независимость потоков: для этого добавим sleep(10) для одной из операций. При вычислении этой операции данный поток ждет 10 секунд, другой, не "спит", работает правильно.

При вводе неверных данных (к примеру: abc), приложение также "падает", т.к. данные не соответствуют протоколу.
\chapter{Реализация для работы по протоколу UDP}

\section{Прикладной протокол}
Прикладной протокол UDP ничем не отличается от TCP, т.к. также передается строка с арифметическим выражением (см.\ref{protocol_tcp}, описывающий протокол для взаимодействия по TCP). 

\section{Архитектура приложения и Тестирование}
Архитектура приложения и тестирование такие же как у TCP, единственное различие - реализация многопоточности. Так как в данной реализации не предусматривается установления логического соединения, то удобна организация асинхронного сервера, реагирующего на сообщения.
Поэтому реализация такой задачи была осуществлена с помощью логики, а не потоков.

\chapter{Выводы}
\section{TCP}
TCP – протокол с обеспечением надежности передачи. TCP гарантирует, что данные не потеряются в пути, придут в правильном порядке и не придут дважды. Однако, так как TCP – потоковый протокол, может возникнуть проблема, связанная с вызовом read/recv , т.к. он может считать лишь часть сообщения. Таким образом, для чтения одного сообщения может понадобиться несколько вызовов read/recv. Поэтому в данном приложении длина сообщения ограниченна. Также протокол ТСР требует, чтобы все отправленные сегменты данных были подтверждены с приёмного конца, т.е. используется алгоритм обратной связи. 
\section{UDP}
    Протокол UDP называют протоколом ненадёжной доставки. Протокол UDP обеспечивает только доставку дейтаграммы и не гарантирует её выполнение. При обнаружении ошибки дейтаграмма просто стирается. Протокол не поддерживает виртуального соединения с удалённым модулем UDP. Чаще всего базируется на принципах динамической маршрутизации (каждая дейтаграмма передаётся по оптимальному маршруту). Основное достоинство — простота.

\chapter*{Приложения}
\section*{Описание среды разработки}
Linux debian 3.2.0-4-486 1 Debian 3.2.60-1+deb7u3 i686 GNU/Linux.\newline Среда разработки - Eclipse.\newline
Windows 7.\newline Среда разработки - Visual Studio 2010.
\section*{Листинги}
\subsection*{TCP сервер}
\lstinputlisting[]
{/home/user/workspace/Simple_tcp_server/main.c}
\subsection*{Файл сборки Makefile}
\lstinputlisting[language=make,label={Makefile}]
{/home/user/workspace/Simple_tcp_server/Makefile}
\subsection*{TCP клиент}
\lstinputlisting[]
{/home/user/workspace/Client/client.c}
\subsection*{TCP клиент Windows}
\lstinputlisting[]
{/home/user/NETS/1.c}
\subsection*{UDP сервер Windows}
\lstinputlisting[]
{/home/user/NETS/2.c}
\subsection*{UDP клиент}
\lstinputlisting[]
{/home/user/workspace/udp_client/client.c}
\end{document}