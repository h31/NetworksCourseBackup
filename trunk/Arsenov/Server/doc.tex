\documentclass[12pt,a4paper]{report}
\usepackage[utf8]{inputenc}
\usepackage[russian]{babel}
\usepackage[OT1]{fontenc}
\usepackage{amsmath}
\usepackage{amsfonts}
\usepackage{amssymb}
\usepackage{graphicx}
\begin{document}
\section{Протокол}
Клиент отправляет команды в виде строки с текстом, заканчивающейся символом перевода строки (\textbackslash n).
\begin{itemize}
\item show index - показать все доступные в магазине товары
\item show cart - показать содержимое корзины
\item buy N - купить товар под номером N
\item add N P C - добавить в каталог товар с именем N по цене P в количестве C
\end{itemize}
Формат ответа на команды show cart и show index:

\begin{verbatim}
I N P C
\end{verbatim}

Где I - номер товара в магазине, N - его название, P - его цена, C - количество.

Каждый товар выводится на отдельной строке.
\end{document}
