\documentclass[10pt,a4paper]{report}
\usepackage[utf8]{inputenc}
\usepackage[russian]{babel}
\usepackage[OT1]{fontenc}
\usepackage{amsmath}
\usepackage{amsfonts}
\usepackage{amssymb}
\usepackage{graphicx}
\author{Баратынский А.И.}
\title{Отчет по лабораторной работе на тему "Ознакомление с системой верстки \TeX\ и расширением \LaTeX "\newline Дисциплина: "Телекоммуникационные системы и сети"}
\date{Март 2014}
\begin{document}
\maketitle
\pagebreak
\chapter{Ознакомление}
\section{Цель работы}
Изучение принципов работы \TeX и создание первого отчета
\section{Написание формул}
Задача - научиться писать формулы в системе. 
\chapter{Ход работы}
\section{Написание формул}
\begin{displaymath}
X(t) = A_0 * (sin(2*t\sqrt\pi))
\end{displaymath}
\begin{displaymath}
f(x) = \frac{A_0}{2} + \sum \limits_{n=0}^{\infty} A_n \sin \left( \frac{2 n \pi^2 x}{\nu} - \alpha_n \right)
\end{displaymath}
\begin{displaymath}
c^2 = a^2 + b^2 \frac{a+b}{c-d} \sum_{i=0}^{N}k _i
\end{displaymath}
\begin{displaymath}
dz = \frac{\partial z}{\partial x} dx + \frac{\partial z}{\partial y} dy
\end{displaymath}
\begin{displaymath}
f(x,y,\alpha, \beta) = \frac{\sum \limits_{n=0}^{\infty}
A_n \sin \left( \frac{3 n \pi^2 x}{\nu} \right)} {\prod \mathcal{F} {g(x,y)} }-\int{x} f(x) dx
\end{displaymath}
\section{Вывод}
В данной работе мы впервые работали с системой верстки \TeX\ и расширением \LaTeX. На мой взгляд, это очень удобное средство для создания отчетов и других текстовых документов. 
\end{document}
