\documentclass[10pt,a4paper]{report}
\usepackage[utf8]{inputenc}
\usepackage[russian]{babel}
\usepackage[OT1]{fontenc}
\usepackage{amsmath}
\usepackage{amsfonts}
\usepackage{amssymb}
\usepackage{graphicx}
\author{Скрипаль Борис}
\title{Отчет по лабораторной работе по дисциплине "Сети и системы передачи данных"\newline
на тему "Визуализация сигналов во временной и частотной области"}
\date{23.02.14}
\begin{document}
\maketitle
\pagebreak
\chapter{Теоретическая часть}
\section{Цель работы}
Познакомиться со средствами генерации сигналов и визуализации их спектров.
\section{Постановка задачи}
В командном окне MATLAB и в среде Simulink промоделировать чистый синусоидальный сигнал, 
так же синусоидальный сигнал с шумом. Получить их спектры.
\section{Введение}
В ходе данной лабораторной работы необходимо промоделировать чистый синусоидальный сигнал, а так же синусоидальный сигнал с шумом и получить их представления во временной и частотной областях. Синусоидальный сигнал задаётся по следующей формуле: 
\begin{displaymath}
A(t) = A_0 * sin(2*\pi *f*t + u_0)
\end{displaymath}
Для создания зашумленного синусоидального сигнала, к чистому синусоидальному сигналу прибавляется случайная составляющая, по формуле:
\begin{displaymath}
A(t) = A_0 * sin(2*\pi *f*t + u_0) + A_1*rand()
\end{displaymath}
Для выделения частот регулярных составляющих сигнала необходимо использовать преобразование Фурье, реализуемое следующей формулой:
\begin{displaymath}
X(k) = \sum_{j=1}^N x(j)*e^{2*\frac{x}{N(j-1)(k-1)}}
\end{displaymath}
\chapter{Ход работы}
\section{Алгоритм работы}
\begin{itemize}
\item Построение чистого синусоидального сигнала с регулярной составляющей 10 Гц и с нулевой начальной фазой
\item Вывод временной характеристики сигнала
\item Реализация одномерного преобразования Фурье на основе 512 точек
\item Построение графика спектральной плотности для чистого синусоидального сигнала
\item Построение зашумленного синусоидального сигнала путем добавления к чистому синусоидальному сигналу случайной аддитивной компоненты с нулевым средним
\item Вывод временной характеристики полученного сигнала
\item Реализазия одномерного преобразования Фурье на основе 512 точек
\item Построение графика спектральной плотности для зашумленного синусоидального сигнала
\end{itemize}
\end{document}