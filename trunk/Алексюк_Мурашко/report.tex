\documentclass[10pt,a4paper]{article}
\usepackage{cmap}
\usepackage[OT1]{fontenc}
\usepackage[utf8]{inputenc}
\usepackage[russian]{babel}
\usepackage{amsmath}
\usepackage{amsfonts}
\usepackage{amssymb}
\usepackage{graphicx}
\author{Алексюк А.О., Мурашко Д.С.}
\title{Cистема верстки \TeX и расширения \LaTeX}
\begin{document}
\maketitle
\tableofcontents
\pagebreak
\section{Система верстки \TeX~ и расширение \LaTeX}
\subsection{Цель работы}
Изучение принципов работы \TeX и создание первого отчета.
\subsection{Ход работы}
\paragraph{Изучение}
\begin{enumerate}
\item блаблабла
\item мда мда мда
\end{enumerate}
$$\int\limits_2^{\infty} blablabla$$
\\
$$D = \frac{n^2}{\phi}$$
\hspace{10cm} $c^2=a^2+b^2$
$\Omega + 
\omega$
\paragraph{Вывод}
\LaTeX{} --- довольно удобный инструмент для создания документов. Из положительных сторон можно отметить простой набор формул и красивый шрифт по умолчанию. Оформление текста с помощью команд проблем не вызвало.
\end{document}