\documentclass[10pt,a4paper]{article}
\usepackage{cmap}

\usepackage[OT1]{fontenc}
\usepackage[utf8]{inputenc}
\usepackage[russian]{babel}
\usepackage{amsmath}
\usepackage{amsfonts}
\usepackage{amssymb}
\usepackage{graphicx}
\author{Алексюк А.О., Мурашко Д.С.}
\title{Cистема верстки \TeX и расширения \LaTeX}
\begin{document}
\maketitle
\tableofcontents
\pagebreak
\section{Система верстки \TeX~ и расширение \LaTeX}
\subsection{Цель работы}
Изучение принципов работы \TeX и создание первого отчета.
\subsection{Ход работы}
\paragraph{Нумерованный список}
\begin{enumerate}
\item Элемент списка
\item блаблабла
\end{enumerate}
\paragraph{Формулы}
$$\int\limits_2^{\infty} blablabla$$
\\
$$D = \frac{n^2}{\phi}$$
\hspace{10cm} $c^2=a^2+b^2$
$\Omega + 
\omega$
\begin{itemize}
\item $\sum\limits_{n=0}^{N-1} a_0 q^n = a_0 \dfrac{q^N-1}{q-1}$

\item $\Phi_\text{ст} = \Phi_\text{д}(f) \dfrac{1}{T_2} \Phi_\text{П}(f) = \dfrac{\sin(\pi f T_2)}{\pi f T_2} \sum\limits_{k=-\infty}^{\infty} \Phi (f - k f_2)$
\end{itemize}
\subsection{Вывод}
\LaTeX{} --- довольно удобный инструмент для создания документов. Из положительных сторон можно отметить простой набор формул и красивый шрифт по умолчанию. Проблем с синтаксисом \LaTeX{} не возникло.
\end{document}