\chapter{Частотная и фазовая модуляция}
\section{Постановка задачи}
\begin{enumerate}
\item 
Сгенерировать однотональный сигнал низкой частоты.
\item
Выполнитьфазовую модуляцию/демодуляцию сигнала по закону $u(t)=(U_m\cos(\omega t=ks(t))$ , используя встроенную функцию MatLab pmmod, pmdemod.
\item
Получить спектр модулированного сигнала.
\item
Выполнить частотную модуляцию/демодуляцию по закону 
\begin{equation}
u(t)=U_m\cos (\omega _0 t+k \int_0^t s(t)dt=\phi _0)
\end{equation}
используя встроенные функции Matlab fmmod, fmdemod.
\end{enumerate}
\section{Теоретическая часть}
Частотная модуляция — вид аналоговой модуляции, при котором информационный сигнал управляет частотой несущего колебания. По сравнению с амплитудной модуляцией здесь амплитуда остаётся постоянной.

Фазовая модуляция — один из видов модуляции колебаний, при которой фаза несущего колебания управляется информационным сигналом. Фазомодулированный сигнал s(t) имеет следующий вид:
\begin{equation} 
s(t) = g(t) \sin[2 \pi f_c t + \varphi(t)] ,
\end{equation}
где $g(t)$ — огибающая сигнала; $\phi(t)$ является модулирующим сигналом; $f_c$ — частота несущего сигнала; t — время.\\
По характеристикам фазовая модуляция близка к частотной модуляции. В случае синусоидального модулирующего (информационного) сигнала, результаты частотной и фазовой модуляции совпадают.
\section{Ход работы}
\subsection{Код Matlab}
\begin{lstlisting}
x = 0:0.01:2;
u = cos(2*pi*x);

figure;
subplot(3,1,1);
plot(x,u);
grid;
am = fmmod(u,2,30,1);
subplot(3,1,2);
plot(x,am);
grid;
modulatedSpectr = fft(am,512);
normSpectrum = modulatedSpectr.*conj(modulatedSpectr)/512;
f = 100*(-256:255)/512;
subplot(3,1,3);
plot(f,normSpectrum);
grid;
axis([min(f) max(f) 0 max(normSpectrum)]);

figure;
subplot(3,1,1);
plot(x,u);
grid;
am = pmmod(u,5,1000,pi/2);
subplot(3,1,2);
plot(x,am);
grid;
modulatedSpectr = fft(am,512);
normSpectrum = modulatedSpectr.*conj(modulatedSpectr)/512;
f = 100*(-256:255)/512;
subplot(3,1,3);
plot(f,normSpectrum);
grid;
axis([min(f) max(f) 0 max(normSpectrum)]);
\end{lstlisting}
\section{Результаты работы}


\begin{figure}[H]

\includegraphics[width=150mm, scale = 0.9]{lab8/8_1}
   \caption{Частотная модуляция сигнала}

\end{figure}
\begin{figure}[H]

\includegraphics[width=150mm, scale = 0.9]{lab8/8_2}
   \caption{Фазовая модуляция сигнала}

\end{figure}
Смоделируем ход работы в среде Simulink:
\begin{figure}[H]

\includegraphics[width=150mm, scale = 0.9]{lab8/8_3}
   \caption{Модель для частотной модуляции}

\end{figure}
\begin{figure}[H]

\includegraphics[width=150mm, scale = 0.9]{lab8/8_4}
   \caption{Исходный сигнал}

\end{figure}
\begin{figure}[H]

\includegraphics[width=150mm, scale = 0.9]{lab8/8_5}
   \caption{Моделированный сигнал}

\end{figure}
\begin{figure}[H]

\includegraphics[width=150mm, scale = 0.9]{lab8/8_6}
   \caption{Спектр моделированного сигнала}

\end{figure}
\begin{figure}[H]

\includegraphics[width=150mm, scale = 0.9]{lab8/8_7}
   \caption{Модель для фазовой модуляции}

\end{figure}
\begin{figure}[H]

\includegraphics[width=150mm, scale = 0.9]{lab8/8_8}
   \caption{Исходный сигнал}

\end{figure}
\begin{figure}[H]

\includegraphics[width=150mm, scale = 0.9]{lab8/8_9}
   \caption{Моделированный сигнал}

\end{figure}
\begin{figure}[H]

\includegraphics[width=150mm, scale = 0.9]{lab8/8_10}
   \caption{Спектр моделированного сигнала}

\end{figure}
Выполним демодуляцию, в том числе с помощью блока захвата фазы (фазовой автоподстройки частоты) Phase-Locked Loop.
\begin{figure}[H]

\includegraphics[width=150mm, scale = 0.9]{lab8/8_11}
   \caption{Модель фазовой автоподстройки часоты в режиме слежения}

\end{figure}
\begin{figure}[H]

\includegraphics[width=150mm, scale = 0.9]{lab8/8_12}
   \caption{Модель фазовой автоподстройки часоты в режиме захвата и удержания сигнала}

\end{figure}
\begin{figure}[H]

\includegraphics[width=150mm, scale = 0.9]{lab8/8_13}
   \caption{Исходный сигнал}

\end{figure}
\begin{figure}[H]

\includegraphics[width=150mm, scale = 0.9]{lab8/8_14}
   \caption{Моделированный сигнал}

\end{figure}
\begin{figure}[H]

\includegraphics[width=150mm, scale = 0.9]{lab8/8_15}
   \caption{Сигнал на выходе ФНЧ}

\end{figure}
\begin{figure}[H]

\includegraphics[width=150mm, scale = 0.9]{lab8/8_16}
   \caption{Сигнал на выходе фазового детектора}

\end{figure}
\begin{figure}[H]

\includegraphics[width=150mm, scale = 0.9]{lab8/8_17}
   \caption{Сигнал на выходе ГУН}

\end{figure}
\section{Вывод}

В результате выполнения данной работы были выполнены частотная и фазовая модуляция/демодуляция, а также чатотная демодуляция с  помощью блока захвата фазы. Можно сделать вывод, что частотная и фазовая модуляция очень тесно взаимосвязаны, поскольку обе они влияют на аргумент функции cos. Поэтому эти два вида модуляции имеют общее название — угловая модуляция.Сигнал с угловой модуляцией имеет вид колебания, начальная фаза которого зависит от времени:
	\begin{equation}
	s_(t) = A_0 cos(\omega0 t + j(t)).
	\end{equation}
Различие между фазовой и частотной модуляцией заключается лишь в том, как именно начальная фаза $j(t)$ связана с модулирующим сигналом.

Для демодуляции использовалась петля ФАПЧ, состоящая из перемножителя (используемого в качестве фазового детектора), фильтра нижних частот и генератора, управляемого напряжением (ГУН). Получаемый на выходе петли ФАПЧ сигнал пропорционален отклонению мгновенной частоты модулированного сигнала от несущей частоты, поэтому при демодуляции ФМ этот сигнал можно дополнительно проинтегрировать, чтобы получить начальную фазу сигнала.