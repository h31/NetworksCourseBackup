\chapter{Система верстки \TeX~ и расширение \LaTeX}
\section{Цель работы}
Изучение принципов работы \TeX и создание первого отчета.
\section{Ход работы}
\paragraph{Создание документа}

Для создания нового документа проще всего воспользоваться инструментом "Быстрый старт" в TeXMaker. С помощью этого инструмента можно выбрать класс документа, размер бумаги, кодировку и т.д. С тем же успехом можно создать простой текстовый файл и вручную задать эти параметры с помощью директив documentclass и usepackage.

\paragraph{Компиляция в командной строке}

Для сборки документа нужно выполнить команду latex report.tex, где report.tex - путь к собираемому файлу. В результате сборки появится файл с расширением dvi, который можно просмотреть, например, в xdvi. Для получения PDF необходимо либо воспользоваться dvips, а затем ghostscript, либо сразу запускать pdflatex вместо latex.

То же самое можно сделать прямо в TeXMaker через меню "Инструменты".

\paragraph{Нумерованный список}
\begin{enumerate}
\item Элемент списка
\item блаблабла
\end{enumerate}
\paragraph{Формулы}
$$\int\limits_2^{\infty} blablabla$$
\\
$$D = \frac{n^2}{\phi}$$
\hspace{10cm} $c^2=a^2+b^2$
$\Omega + 
\omega$
\begin{itemize}
\item $\sum\limits_{n=0}^{N-1} a_0 q^n = a_0 \dfrac{q^N-1}{q-1}$

\item $\Phi_\text{ст} = \Phi_\text{д}(f) \dfrac{1}{T_2} \Phi_\text{П}(f) = \dfrac{\sin(\pi f T_2)}{\pi f T_2} \sum\limits_{k=-\infty}^{\infty} \Phi (f - k f_2)$
\end{itemize}
\section{Вывод}
\LaTeX{} --- довольно удобный инструмент для создания документов. Из положительных сторон можно отметить простой набор формул и красивый шрифт по умолчанию. Проблем с синтаксисом \LaTeX{} не возникло.