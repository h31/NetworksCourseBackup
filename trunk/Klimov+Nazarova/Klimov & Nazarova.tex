\documentclass[10pt,a4paper]{article}
\usepackage[utf8]{inputenc}
\usepackage[russian]{babel}
\usepackage[OT1]{fontenc}
\usepackage{amsmath}
\usepackage{amsfonts}
\usepackage{amssymb}
\usepackage{graphicx}
\author{Климов С.А., Назарова К.Е.}
\title{Отчет по лабораторным работам по дисциплине ТСС}
\date{2014}
\begin{document}
\maketitle
\tableofcontents
\pagebreak
\section{Система верстки \TeX и расширения \LaTeX}
\subsection{Цель работы}
Изучение принципов верстки ТеХ, создание первого отчета
\subsection{Ход работы}
\paragraph{Изучение}
\begin{enumerate}
\item Создание минимального файла .tex в простом текстовом редакторе - преамбула, тело документа
\item Компиляция в командной строке - latex, xdvi, pdflatex
\item Оболочка TexMaker, Быстрый старт, быстрая сборка
\item Создание титульного листа, нескольких разделов, списка, несложной формулы
\end{enumerate}
\paragraph{Выполнение практического задания}
\begin{displaymath}
c^2 = a^2 + b^2
\end{displaymath}
\begin{equation}
c_1+c_2=b_0\frac{2*a}{\log2}
\sum_{i=0}^{\infty}\Theta
\end{equation}
\section{Визуализация сигналов во временной и частотной области}
\end{document}